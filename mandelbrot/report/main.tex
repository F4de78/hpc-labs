\documentclass{article}

\usepackage{natbib} % Required to change bibliography style to APA
\usepackage{amsmath} % Required for some math elements 
\usepackage{geometry}
\usepackage[english]{babel}
\usepackage{float}
\usepackage {tikz}
\usepackage{graphicx}
\usepackage{comment}
\usepackage[export]{adjustbox}
\usetikzlibrary {positioning}
\graphicspath{ {./images/} }
\usepackage{hyperref}
\usepackage{multicol}
\usepackage{fancyvrb}
\usepackage{xcolor}
\usepackage[utf8]{inputenc}
\usepackage{fancyvrb}
\usepackage{xcolor}
\usepackage{animate}
\usepackage{array}
\usepackage{minted}
\usepackage{subfig}
\usepackage{booktabs}
\usepackage{mdframed}
\usepackage{drawstack}

\usemintedstyle{perldoc}

\definecolor{bg}{rgb}{0.95,0.95,0.95}
\definecolor{dark}{rgb}{0.25,0.25,0.25}

\BeforeBeginEnvironment{verbatim}{\begin{mdframed}[backgroundcolor=bg]}}
\AfterEndEnvironment{verbatim}{\end{mdframed}}
\BeforeBeginEnvironment{minted}{\begin{mdframed}[backgroundcolor=bg]}}
\AfterEndEnvironment{minted}{\end{mdframed}}



\newcommand{\ip}[1]{{\fontfamily{pcr}\selectfont #1}}
\definecolor{LightGray}{gray}{0.9}

\parskip=0pt plus 1pt
\parindent=15pt

\setlength\parskip{1em plus 0.1em minus 0.2em}
\setlength\parindent{0pt}

 \geometry{
 a4paper,
 total={170mm,257mm},
 left=22mm,
 top=22mm,
 }




\title{High Performance Computing \\ Final project \\ Mandelbrot  }
\author{
\begin{tabular}[t]{c@{\extracolsep{3em}}c@{\extracolsep{3em}}c} 
Federico Fontana  & Filippo Siri & Marco Zucca \\
s4835118@studenti.unige.it & s4819642@studenti.unige.it & s4828628@studenti.unige.it 
\end{tabular}
}
\date{March 2024}

\begin{document}

\maketitle

\tableofcontents
\newpage

\section{Introduction}
This project aimed to optimise an implementation of an algorithm that generates the Mandelbrot set. To do so, we write a version of the algorithm to be executed by the GPU and other versions that leverage multi-processing and vectorization. We also added other optimisations to reduce intensive computation.

\section{Setup}
We run our implementation on a machine equipped with a 12th Gen Intel(R) Core(TM) i9-12900K CPU and an NVIDIA T400 GPU. \\

The CPU has 24 core, 8 `e-core’ (no hyperthreading) and 8 `p-core’ (hyperthreading factor 2) and support AVX2 extension.

\section{Hotspots}
Before optimising the algorithm, we started looking at the code to identify where most of the computing is done. \\
Since the code is quite simple, it is easy to identify the following hotspot:

\begin{minted}{c}
for (int pos = 0; pos < HEIGHT * WIDTH; pos++)
    {
        image[pos] = 0;

        const int row = pos / WIDTH;
        const int col = pos % WIDTH;
        const complex<double> c(col * STEP + MIN_X, row * STEP + MIN_Y);

        // z = z^2 + c
        complex<double> z(0, 0);
        for (int i = 1; i <= ITERATIONS; i++)
        {
            z = pow(z, 2) + c;

            // If it is convergent
            if (abs(z) >= 2)
            {
                image[pos] = i;
                break;
            }
        }
    }
\end{minted}
It can be seen that while there are no dependencies in the outermost loop, the calculations performed in the innermost loop depend on the previous iteration.
%TODO: usare intel advx

\section{CPU}
\subsection{Parallelisation}
%We parallelised the code using OpenMP APIs, by adding a directive above the first loop, that 
We parallelized the outermost loop because, as mentioned in the previous section, it was the only loop where there were no dependencies.\\
To do this, it was sufficient to add an instruction provided by the OpenMP APIs that allows to specify the type of schedule and number of threads to use.\\
We can see the implementation in the following code:
\begin{minted}{c}
#pragma omp parallel for schedule(OMP_SCHEDULE) num_threads(THREAD_NO)
for (int pos = 0; pos < HEIGHT * WIDTH; pos++)
{
    image[pos] = 0;

    const int row = pos / WIDTH;
    const int col = pos % WIDTH;
    const complex<double> c(col * STEP + MIN_X, row * STEP + MIN_Y);

    // z = z^2 + c
    complex<double> z(0, 0);
    for (int i = 1; i <= ITERATIONS; i++)
    {
        z = pow(z, 2) + c;

        // If it is convergent
        if (abs(z) >= 2)
        {
            image[pos] = i;
            break;
        }
    }
}
\end{minted}

\subsection{Vectorization}
The intel compiler cannot vectorise the innermost loop because of the dependency seen above. \\ 
Although no dependency is present, the outermost loop cannot be vectorised automatically using the Intel compiler either due to the presence of the \texttt{break} because this makes the number of iterations non-constant and varies according to a condition verified at runtime. \\
Nevertheless, it is still possible to vectorize, by using Intel intrinsics because allows to manage the internal break manually. \\ 
As said before, the CPU supports AVX2, so we can use at most 256-bit long vectors and leverage AVX2-specific instructions (wrapped by the intrinsics), like three operand fused-multiply-add (FMA3).



Nell'implementazione effettuata utilizzando gli intrinsics abbiamo cambiato il ciclo esterno smettendo di iterare sulle singole celle, ma iterando su blocchi di celle che verranno vettorizzate insieme (4 celle utilizzando i double ed 8 celle utilizzando i float). \\
Utilizzando la vettorizzazione abbiamo anche dovuto riscrivere le operazioni sui numeri complessi senza utilizzare il tipo complex. \\
Infine abbiamo dovuto cambiare la condizione di uscita dal ciclo più interno ed il salvataggio dei valori nell'output. Si è trattata della parte più complicata perchè abbiamo dovuto proprio cambiare la logica utilizzata. \\
Per uscire dal ciclo abbiamo creato una maschera dove veniva salvato 0 in ogni bit che rappresentava una determinata cella in un blocco quando il valore assoluto non era ancora maggiore di 4 ed 1 quando invece lo era, successivamente abbiamo trasformato questa maschera in un'interno composto nel seguente modo:
\begin{minted}{c}
FOR j := 0 to 3
    i := j*64
    IF a[i+63]
	   dst[j] := 1
    ELSE
	   dst[j] := 0
    FI
ENDFOR
dst[MAX:4] := 0
\end{minted}
Per salvare il valore solamente una volta abbiamo aggiunto una nuova variabile \texttt{mask} che permette di indicare se il valore di una cella presente nel blocco è stato salvato o no. \\
Questa variabile è stata inizializzata con tutti i bit ad 1 e quando veniva salvato il valore di una cella i bit della maschera vengono messi a 0. Per farlo abbiamo utilizzato la seguente logica, dove indichiamo con pos i bit che fanno riferimento ad una determinata cella:
\begin{minted}{c}
mask[pos] = mask[pos] && (mask[pos] ^ abs2_gt_4[pos]) 
\end{minted}
Lo xor interno porta a 0 i bit di mask[pos] quando per la prima volta il valore di quella cella diventa maggiore di 4, l'and con il vecchio valore di mask[pos] permette di evitare che i bit di mask[pos] tornino ad 1 nell'iterazione successiva. \\
Per salvare il valore dentro l'output abbiamo utilizzato la seguente logica:
\begin{minted}{c}
image[pos] = image[pos] || (abs2_gt_4[pos] && mask[pos] && current_step[pos])
\end{minted}
La seconda parte dell'or restituisce i bit che rappresentano lo step corrente solamente quando tutti quel valore è maggiore di 4 (abs2\_gt\_4[pos] contiene tutti 1) e non è ancora stato salvato un valore per quella cella (mask[pos] contiene tutti 1). Siccome la seconda parte dell'or restituisce 0 quando il valore è già stato salvato, per evitare di perdere il valore precedente, il valore viene messo in or con il valore precedente (possibile perchè all'inizio image[pos] contiene tutti i bit a 0).
\subsection{Results}

\section{GPU}

\subsection{Results}


\end{document}